\documentclass[11pt]{article}
\usepackage{amsmath,amsthm,amsfonts}
\usepackage{color}
\usepackage{ amssymb }
\usepackage{hyperref}
\numberwithin{equation}{section} \allowdisplaybreaks
\newcommand{\R}{\mathbb R}
\newcommand{\C}{\mathbb C}
\newcommand{\N}{\mathbb N}

\textwidth=140truemm \textheight=200truemm \oddsidemargin=1truecm
\evensidemargin=1truecm

\newtheorem{theorem}{\sc Theorem}[section]
\newtheorem{lemma}[theorem]{\sc Lemma}
\newtheorem{proposition}[theorem]{\sc Proposition}
\newtheorem{corollary}[theorem]{\sc Corollary}
\newtheorem{definition}[theorem]{\sc Definition}
\newtheorem{example}[theorem]{\sc Example}
\newtheorem{remark}[theorem]{\sc Remark}
\newtheorem{remarks}[theorem]{\sc Remarks}
\newtheorem{examples}[theorem]{\sc  Examples}
\newtheorem{Application}[theorem]{\sc  Application}
\newcommand{\bremarks}{\begin{remarks}\rm}
\newcommand{\eremarks}{\end{remarks}}
\newcommand{\bexamples}{\begin{examples}\rm}
\newcommand{\eexamples}{\end{examples}}
\newcommand{\bapp}{\begin{Application}\rm}
\newcommand{\eapp}{\end{Application}}

\def \n {\noindent}
\def \n {\noindent}
\def \no {\noindent}
\renewcommand{\proof}{\noindent{\it Proof.}\enspace}
\newcommand{\bet}{\begin{theorem}}
\newcommand{\eet}{\end{theorem}}
\newcommand{\blm}{\begin{lemma}}
\newcommand{\elm}{\end{lemma}}
\newcommand{\bprop}{\begin{proposition}}
\newcommand{\eprop}{\end{proposition}}
\newcommand{\bcor}{\begin{corollary}}
\newcommand{\ecor}{\end{corollary}}
\newcommand{\bdf}{\begin{definition}\rm}
\newcommand{\edf}{\end{definition}}
\newcommand{\bp}{\begin{proof}}
\newcommand{\ep}{\end{proof}}
\newcommand{\bex}{\begin{example}\rm}
\newcommand{\eex}{\end{example}}
\newcommand{\bremark}{\begin{remark}\rm}
\newcommand{\eremark}{\end{remark}}


\newcommand{\nul}[1]{\mathrm{N}( {#1} )}
\newcommand{\ran}[1]{\mathrm{R}({#1})}
\newcommand{\nulbig}[1]{\mathrm{N}\left( {#1} \right)}
\newcommand{\ranbig}[1]{\mathrm{R}\left({#1} \right)}

\newcommand{\norm}[1]{ \| #1 \| }
\newcommand{\diag}[2]{\mathrm{diag}(#1 \,,\, #2) }

\def\NN{{\mathbb N}}

\newcommand{\wsupp}[1]{{#1}^{\mathrm{\sigma,W}}}
\newcommand{\wdraz}[1]{{#1}^{\mathrm{D,W}}}
\newcommand{\proj}[1]{{#1}^\pi}
\newcommand{\ind}[1]{\mathrm{ind}(#1)}
\newcommand{\mpinv}[1]{{#1}^{\mathrm{\dag }}}
\newcommand{\codim}{\mathrm{codim \,}}
\newcommand{\dis}[1]{\mathrm{dis }(#1)}

\newcommand{\bx}{{ \mathbf B}(X)}

\newcommand{\ch}{{ \mathcal C}(X)}
\newcommand{\bxy}{{ \mathbf B}(X, Y)}
\newcommand{\qf}{{ \mathbf QF}(H)}
\newcommand{\qfd}[1]{\mathrm{q \Phi}({#1})}
%\newcommand{\bfred}{BF(H)}

%%%%%%%%%%%%%%%%%%%%

\pagestyle{myheadings}\markboth{\rm M. Berkani }%
{\hss\it  Berkani: Discrete spectral theorem}
%%%%%%%%%%%%%%%%%%%%

\begin{document}

\title {  On the discrete spectral theorem }

\author { M. Berkani }


\date{}

\maketitle


\begin{abstract}




{\tiny In this paper,
 give a necessary and sufficient condition for a   bounded linear
meromorphic operator with infinite spectrum and simple poles, to
have a  spectral decomposition in terms of its poles and the
corresponding spectral projections, obtaining thereby  a
``discrete spectral theorem'' for  meromorphic operators which
extends the usual spectral decomposition theorem of normal compact
operators.}

\end{abstract}

\renewcommand{\thefootnote}{}

\footnotetext{\hspace{-7pt}2010 {\em Mathematics Subject
Classification\/}:  primary 47A10, 47A53.
\baselineskip=18pt\newline\indent {\em Key words and phrases\/}:
     meromorphic,  decomposition, discrete spectral theorem, totally hereditarily normaloid }

\tableofcontents





\section{Introduction}

   Let $\ch$ be  the  set of  linear closed operators defined
from a Banach space  $X$ to $X$ and $L(X)$ be the Banach algebra
of  bounded linear  operators defined from   $X$ to $X.$ We write
${\mathrm D}(T)$, $\nul{T}$ and $\ran{T}$ for the domain,
nullspace and range of an operator $T\in \ch$. An operator $T\in
\ch$  is called a {\it Fredholm}  operator \cite{TLY} if both the
nullity  $n(T)=\dim \nul{T}$ of $T$ and the defect $d(T)=\codim
\ran{T}$  of $T$  are finite. The index $i(T)$ of a Fredholm
operator $T$ is defined by $i(T)=n(T)-d(T)$. It is well known that
if $T$ is a Fredholm operator, then $\ran{T}$ is closed.

The class of bounded linear  B-Fredholm operators, which is a
natural extension of the class of Fredholm operators  was
introduced in \cite{P7}, and the class of unbounded linear closed
B-Fredholm operators acting on a Banach space was studied in
\cite{P33}.


Recall \cite{CA-1} that a bounded linear  operator $T$  is called
a meromorphic operator if $ \lambda = 0$ is the only possible
point of accumulation of its spectrum $\sigma(T)$ and every
non-zero isolated point of $\sigma(T)$ is a pole of the resolvent
$R_\texttt{{\tiny $\mu $}}(T)= (T-\mu I)^{-1}$  of $T,$ which is
defined on the resolvent set $\rho(T)$ of $T.$  If we also require
that each non-zero eigenvalue of T has finite multiplicity, then
$T$ will be called a Riesz operator.


A first result linking bounded B-Fredholm operators to the class $
\mathfrak{M}$ of linear bounded meromorphics operators comes from
the following theorem, established in \cite[Theorem 2.11]{P13}.

\bet \label{thm1.1} Let $ T \in L(X).$  Then    $T$ is a
meromorphic operator if and
 only if  $ \sigma_{BF} (T)= \{ \lambda \in \mathbb{C} \mid  T - \lambda I \text {\,
 is not a B-Fredholm operator } \} \subset \{0\}.$
\eet

Recall that for $T\in \ch$,  its  {\it descent} $\delta(T)$ and
its  {\it ascent} $ a(T) $  are defined by
 $ \delta(T)=\inf  \{n\in \mathbb{N}: R(T^n) = R(T^{n+1})\}$
and
 $ a(T)= \inf \{n\in \mathbb{N} : N(T^{n})=N(T^{n+1})\}$. We set formally $\inf\emptyset =\infty$.

A closed linear operator $T\in \ch$ is said to be {\it Drazin
invertible} if $a(T)$ and $ \delta(T)$ are both finite.  In this
case and  if the resolvent set $\rho(T)$ of $T$  is nonempty, then
$a(T)= \delta(T),  R(T^{a(T)})$ is closed and $ X= R(T^{a(T)})
\oplus  N(T^{a(T)}).$



The {\it Drazin spectrum} of $T$ is defined by: $ \sigma_{\cal
D}(T)= \{\lambda\in\mathbb{C}: T-\lambda
   I \text{ not Drazin invertible} \}.$


The set  of  Browder operators is defined by $\mathcal{B}(X)=\{ T
\in \Phi (X) \mid  a (T)<\infty \,\, \text {and} \,\,\delta(T) <
\infty \}$ and
  the {\it  Browder spectrum} of $T$ is defined by: $ \sigma_{{\cal B}}(T)= \{\lambda\in\mathbb{C}:T-\lambda
   I\notin \mathcal{B}(X)\}.$












\bdf \label{def6}  Let $T \in \ch, $  with non-empty resolvent
set. We will say that $T$ has
 a meromorphic (resp. Riesz or  compact) resolvent if there exists a
scalar $\lambda $ in the resolvent set $ \rho(T)$  of  $T$  such
that $ (T-\lambda I)^{-1}$ is a bounded linear  meromorphic (resp.
Riesz or compact) operator. \edf

 \bremark It's easily seen that if $T$ has meromorphic (resp. Riesz or compact)  resolvent, then
 for all
scalar $\lambda $ in the resolvent set $ \rho(T)$  of  $T,$  $
(T-\lambda I)^{-1}$ is a  bounded linear meromorphic (resp. Riesz
or compact) operator.

 \eremark












 In the second  section,
we consider   a  bounded  linear meromorphic operator $S$  with
infinite spectrum   $  \sigma(T)= \{ \lambda_1,.....,
\lambda_n,....\}$  where the $ \lambda_i $'s are arranged by
decreasing modulus and we assume that each $ \lambda_i, i\geq 1,$
is a simple pole of $S.$ If  $P_n$ is the spectral projection onto
the eigenspace corresponding to the pole $\lambda_n,$  $ \sigma_n=
\sigma(S) \setminus
   \{  \lambda_1,....., \lambda_{n-1}\}, \,  n\geq 2, \,  P_{\sigma_n}$ the
spectral projection associated to the spectral set  $\sigma_n,$
\,$M_{\sigma_n}= R(P_{\sigma_n})$  and $ S_n$ the restriction of
$S$ to the invariant subspace $M_{\sigma_n},$  then  $S =
\displaystyle \sum_{k=1}^{\infty}\lambda_k P_k$ in the Banach
algebra $L(X)$ if and only if $\|S_n P_{\sigma_n}\|$ tends to $0$
as $n$ tends to infinity.

This theorem  which we call the ``discrete spectral
theorem''extends the usual spectral decomposition of normal
compact operators. It extends also the spectral decomposition
theorem  given in \cite[Theorem 54.1]{HEU} and gives a discrete
version of the usual spectral theorem for normal operators
\cite[Theorem 7.3]{TLY}.

Then using using this result, we consider   a closed invertible
operator  $T$ with a meromorphic inverse $S$  and   $
\sigma_{bd}(T)= \{ \lambda_1,....., \lambda_n,....\}$ the
B-discrete spectrum  of  $T,$ where the $\lambda_i$'s are arranged
by increasing modulus. Let $P_n$ be the spectral projection onto
the eigenspace corresponding to the eigenvalue $\lambda_n,$ and
let $ S_n$ be the restriction of $S$ to the invariant subspace
$M_{\sigma_n},$ where here $\sigma_n = \sigma(S) \setminus
   \{  \frac{1}{\lambda_1},....., \frac{1}{\lambda_{n-1}}\}=  \{  \frac{1}{\lambda_n},
    \frac{1}{\lambda_{n+1}},...., 0\}, \,\,  n\geq 2. $
If $\|S_n P_{\sigma_n} \|$ tends to $0$ as $n$ tends to infinity,
then  $ D(T)= \{ \displaystyle \sum_{k=1}^{\infty}
\frac{1}{\lambda_k} P_k (x) \mid x \in E \}$ and $ T(y)=
\displaystyle \sum_{k=1}^{\infty} \lambda_k P_k(y) $ for  $y=
S(x)$ in $D(T)$
 if and only if $\mid \mid x- \displaystyle
\sum_{k=1}^{n-1} P_n(x)\mid \mid $ tends to $0$ as $ n$ tends to
infinity.



\section{Spectral decomposition  of meromorphic operators}

In this section,  we use  the  same  notations as those defined in
the introduction. Our main goal is   to  give a spectral
decomposition theorem first for bounded linear meromorphic
operators, and then for closed operators with a meromorphic
inverse. While the usual spectral theorem for normal operators is
expressed in terms of an integral involving spectral measures
\cite[Theorem 7.3]{TLY}, here we express the operator considered
in a discrete form as the sum of a series involving its  poles
 and the associated projections. This
motivates the appellation ``discrete spectral theorem''.




\bet \label{decompo} Let $S$ be a  bounded  linear meromorphic
operator having an infinite spectrum and simple poles. Then  $S$
has a spectral decomposition $S = \displaystyle
\sum_{k=1}^{\infty}\lambda_k P_k$ in the Banach algebra $L(X)$ if
and only if $\|S_n P_{\sigma_n}\|$ tends to $0$ as $n$ tends to
infinity.

\eet

\bp   If $ S= \displaystyle \sum_{k=1}^{\infty}\lambda_k P_k$ in
the Banach algebra $L(X),$ then for $ x \in X,$ we have $ S(
P_{\sigma_n}(x))= \displaystyle \sum_{k=1}^{\infty}\lambda_k
P_k(P_{\sigma_n}(x))= \sum_{k=n}^{\infty}\lambda_k P_k(x),$
because $P_kP_n= \delta_{nk}P_k,$   where $ \delta_{nk}$ is the
Kronecker symbol. Thus $\| S_nP_{\sigma_n}\| \leq\|
\sum_{k=n}^{\infty}\lambda_k P_k\|, $ and so  $\|
S_nP_{\sigma_n}\|$  tends to $0$ as $n$ tends to infinity because
the series $ \displaystyle \sum_{k=1}^{\infty}\lambda_k P_k $
converges in $L(X).$

Conversely, let $ x \in X,$ then  $ \|(S- \displaystyle
\sum_{k=1}^{n-1} \lambda_k P_k) (x)\|= \| (S( I- \displaystyle
\sum_{k=1}^{n-1} P_k ))(x) \| = \|(SP_{\sigma_n})(x) \|=
\|(S_nP_{\sigma_n})(x) \|$ because $I= P_{\sigma_n} +
\displaystyle \sum_{k=1}^{n-1} P_k.$ Thus $ \| (S-
\sum_{k=1}^{n-1} \lambda_k P_k) (x)\| \leq \|
S_nP_{\sigma_n}\|\,\| x \|$ and so $\| \displaystyle S-
\sum_{k=1}^{n-1} \lambda_k P_k\mid \mid \leq \| S_n
P_{\sigma_n}\|.$ This last inequality establish the converse
implication. \ep

A totally hereditarily normaloid  operator $S$  is an hereditarily
normaloid operator and every invertible part of $S$ is normaloid.


 \bcor  Let $ S$  be a totally hereditarily
 normaloid operator, with  an infinite spectrum $ \sigma(T)= \{ \lambda_1,....., \lambda_n,...., 0\}$
 where the $ \lambda_i $'s are arranged by deacreasing modulus such that $\underset{n \rightarrow \infty} {lim} n\lambda_n= 0. $
Then $S = \displaystyle \sum_{k=1}^{\infty}\lambda_k P_k$ in the
Banach algebra $L(X).$   \ecor

 \bp
 From
 \cite[Theorem 4.3]{DDHZ}, we know that  in this case $S$ is a meromorphic operator  and $ \| P_n\| \leq
 1$  for all  $ n \geq 1.$  Thus if $ n \lambda_n$ tends to $0$ as $n$ tends to
 infinity, we have   $  \|S_nP_{\sigma_n}\|\leq n |\lambda_n|,$   because
$ P_{\sigma_n} = I - \displaystyle \sum_{k=1}^{n-1} P_k.$ From
Theorem  \ref{decompo} we have   $S= \displaystyle
\sum_{k=1}^{\infty} \lambda_kP_k.$

\ep


Let $ (r_n)_n$  be a sequence of positive real numbers such that $
|\lambda_{n+1}| < r_n < |\lambda_{n}| $ and $ \Gamma_n= \{ z \in
\mathbb{C} \mid |z|= r_n\},$ the circle of center $0$ and radius
$r_n.$ Keeping the same notations as before, we have the following
corollary.



\bcor Let $S$ be a  bounded  linear meromorphic operator having an
infinite spectrum and simple poles. If \,\,  $ r_n^2 max \{ \|
(zI- S)^{-1}\| \mid |z|=r_n\}$ tends to $0$ as $n$ tends to
infinity, then $S = \displaystyle \sum_{k=1}^{\infty}\lambda_k
P_k$ in the Banach algebra $L(X).$

\ecor


\bp  We have $ P_{\sigma_n}= \frac{1}{2\pi i}\int_{\Gamma_n}(zI-
S)^{-1} dz.$ From the properties of the functional calculus
\cite[Theorem 48.1]{HEU}, we get  $S_n P_{\sigma_n}= S
P_{\sigma_n} = \frac{1}{2\pi i}\int_{\Gamma_n}z(zI- S)^{-1} dz.$
Under the hypothesis of the corollary, we have $ lim_{n
\rightarrow \infty}\|S_n P_{\sigma_n}\|= 0$  and then Theorem
\ref{decompo} proves the corollary.


\ep


\bcor  Let $ S$  be a Riesz
 operator, with  an infinite spectrum $ \sigma(T)= \{ \lambda_1,....., \lambda_n,...., 0\}$
 where the $ \lambda_i $'s are arranged by deacreasing modulus. If $\|S_n P_{\sigma_n}\|$ tends to $0$ as $n$ tends to
infinity, then $S$ is a compact operator.    \ecor

\bp As  $ lim_{n \rightarrow \infty}\|S_n P_{\sigma_n}\|= 0,$ then
from  Theorem \ref{decompo},    we have   $S = \displaystyle
\sum_{k=1}^{\infty}\lambda_k P_k.$  As $S$ is Riesz operator, then
each $P_k, k\geq 1$ is of finite rank. Thus $S$ is a limit of
finite rank operators. Thus $S$ is compact.

\ep



We use now Theorem \ref{decompo} to  give a ``discrete spectral
theorem''for closed invertible operator with a meromorphic
inverse.


\bet \label{decompo1}
 Let $T$   be a  closed invertible operator having an infinite spectrum and  a meromorphic
inverse  $S$ and   let   $ \sigma_{bd}(T)= \{ \lambda_1,.....,
\lambda_n,....\}$ be the B-discrete spectrum  of  $T,$ where the
$\lambda_i$'s are arranged by increasing modulus. Assume that all
the $\lambda_i's$ are simple poles of $T$  and let $P_n$ be the
spectral projection onto the eigenspaces corresponding to the
eigenvalue $\lambda_n.$ If $S$ has a spectral decomposition $S =
\displaystyle \sum_{k=1}^{\infty}\frac{1}{\lambda_k }P_k,$  then $
D(T)= \{ \displaystyle \sum_{k=1}^{\infty} \frac{1}{\lambda_k} P_k
(x) \mid x \in E \}.$  Moreover  for $y= S(x), x \in X$, we have $
T(y)= \displaystyle \sum_{k=1}^{\infty} \lambda_k P_k(y) $
 if and only if $\mid \mid x- \displaystyle
\sum_{k=1}^{n-1} P_n(x)\mid \mid $ tends to $0$ as $ n$ tends to
infinity. \eet

\bp Observe first that $S$ has simple poles and from
\cite[Proposition 3.3, Proposition 3.4]{P40} we have  $ N(T-
\lambda_n I)= N(S- \frac{1}{\lambda_n} I)$   and $ R(T- \lambda_n
I)= R(S- \frac{1}{\lambda_n} I).$ Thus the projection $P_n$
associated to the eigenvalue $ \lambda_n$ of $T$ is the same as
the projection associated to the eigenvalue $\frac{1}{\lambda_n}$
of $S.$ If $S$ has a spectral decomposition $S= \displaystyle
\sum_{k=1}^{\infty}\frac{1}{\lambda_k} P_k,$ as it  is clear that
$ y \in D(T)$ if and only if there exists $x \in X$ such $S(x)=
y,$ we have  $ y= \sum_{k=1}^{\infty} \frac{1}{\lambda_k} P_k (x)$
and $ D(T)= \{ \displaystyle \sum_{k=1}^{\infty}
\frac{1}{\lambda_k} P_k (x) \mid x \in E \}.$

 Now  assume that  the sequence $(\sum_{k=1}^{n-1} P_k
(x))_n $ tends to $x$ as $n \rightarrow \infty.$ Then $y = lim _{
n \rightarrow \infty} \sum_{k=1}^{n-1} \frac{1}{\lambda_k} P_k
(x)$ and $T( \sum_{k=1}^{n-1} \frac{1}{\lambda_k} P_k (x)) =
\sum_{k=1}^{n-1} P_k (x).$    Since $T$ is closed, then $ T(y)= x=
\displaystyle \sum_{k=1}^{\infty} P_k(x).$ As
$\sum_{k=1}^{\infty}\frac{1}{\lambda_k} P_k(x)= y,$ then $P_k(x)=
\lambda_k P_k(y)$ and $T(y)= \displaystyle \sum_{k=1}^{\infty}
\lambda_k P_k (y).$

Conversely assume that    $T(y)= \displaystyle \sum_{k=1}^{\infty}
\lambda_k P_k (y),$  for  $ y \in D(T).$  Let $x \in X,$ then
  $ x= T(S(x))= \displaystyle
\sum_{k=1}^{\infty} \lambda_k P_k (S(x)).$ Since
$S(x)=\sum_{k=1}^{\infty}\frac{1}{\lambda_k} P_k(x) ,$ then $P_k
(S(x))= \frac{1}{\lambda_k} P_k(x)$ and then  $ x=
\sum_{k=1}^{\infty} P_k (x).$

\ep













\noindent We give now an  example of an operator satisfying the
condition of Theorem \ref{decompo}.



\bexamples

\begin{enumerate}

\item If $S$ is a normal meromorphic operator acting on a Hilbert
space $H,$ then it is well known that all the poles of $S$ are
simple, $ \| S_n \|= |\lambda_n|$ and $\|P_{\sigma_n}\|= 1. $
Since $S$ has an infinite spectrum and is Meromorphic,  then $\|
S_n P_{\sigma_n}\|$ tends to $0$ as $n$ tends to infinity. Thus $S
= \displaystyle \sum_{k=1}^{\infty}\lambda_k P_k.$ In particular
if $S$ is a normal and compact operator, then Theorem
\ref{decompo} gives the well known spectral decomposition theorem
for normal compact operators.



\item If $S$ satisfies  the conditions of \cite[Theorem
54.1]{HEU}, that is  $||S_n||= |\lambda_n|,$ for all $n\geq 1,$
and moreover $S$ satisfies one of the following conditions:

 \hspace{1cm} $\diamond$ The sequence $(\| P_{\sigma_n}\mid \mid)_n$ is bounded.

\hspace{1cm} $\diamond$ $ 2^n \lambda_n   \rightarrow 0,$ as $n
\rightarrow \infty$

\hspace{1cm} $\diamond$ $ \| P_n\mid \mid = 1$ and $ n \lambda_n
\rightarrow  0,$ as $n \rightarrow  \infty$

In this case  $  \|S_nP_{\sigma_n}\| \leq |\lambda_n|
\|P_{\sigma_n}\|$ and from  \cite[Proposition 54.5]{HEU}, we have
$\| P_n\| \leq 2^n.$ Moreover, we know from \cite[ Proposition
54.3]{HEU} that all the poles of $S$ are simple. Then if one of
the previous conditions is satisfied, we have  $ \| S_n
P_{\sigma_n}\| $ tends to $0$ as $n$ tends to infinity and so  the
conclusion of Theorem \ref{decompo} holds.


However,  the following example shows that there exists even Riesz
operators, which do not satisfies  the conclusion of Theorem
\ref{decompo}.


\item \cite[Example 3]{WEST} Consider the operator defined on the
Banach space $ X= l^1(\mathbb{N})$ by :
$S((x_1,x_2,....,x_n,....))= (0, x_1 +\frac{1}{log2} x_2,
\frac{1}{log3} x_3, x_3 + \frac{1}{log4}x_4,  .....).$ Then from
\cite{WEST}, the series $ \displaystyle
\sum_{k=1}^{\infty}\lambda_k P_k$ associated to $S$ is a divergent
series in $L(X),$  thought the operator $S$ is even a Riesz
operator with infinite spectrum  $ \sigma(S)= \{0\} \cup
\{\frac{1}{logj} | j \geq 2\}$ and the poles of $S$  are simple.

\end{enumerate}

\eexamples


\bremarks
\begin{enumerate}

\item  If $S$ is meromorphic with  a finite spectrum $\sigma(S)=
\{ \lambda_1,...., \lambda_N\},$ then from \cite[Theorem 11.3, p.
338]{TLY}, we have    $S = \displaystyle \sum_{k=1}^{N}\lambda_k
P_k.$
%\vspace{0.2cm}

\item  The condition \, $ x= \sum_{k=1}^{\infty} P_k (x)   x \in
X, $ appearing in Theorem \ref{decompo1} could be seen as an
equivalent of the  relation  $ E(\sigma(T))= \int_{\sigma(T)}1 dE
= I$ for the resolution of the identity in the case of spectral
measures \cite[Theorem 7.3]{TLY}.

\item If the operator $T$ considered in Theorem \ref{decompo1} is
not invertible, since the resolvent set of $T$ is nonempty, there
exist $\lambda$ such that $ T-\lambda I$ is invertible. So the
results will be applied to $ T-\lambda I.$






\end{enumerate}

 \eremarks

\section{References}
\begin{thebibliography}{16}

\bibitem{P7}{M. Berkani},
{\em On a class of quasi-Fredholm operators}, Integr. Equ.
Oper.Theory, 34 (1999), 244-249.

\bibitem{P13} {M. Berkani}, {\em B-Weyl spectrum and poles of the
resolvent}, J. Math. Anal. App. 272 (2002), 596-603.

\bibitem{P33}{M. Berkani},
{\em On the B-Fredholm Alternative }, Mediterr. J. Math. ,
10(3),\, {2013},\,1487-1496.

\bibitem{P40}{ M. Berkani, N. Moalla,} {\em B-Fredholm properties of closed invertible operators,}  Mediterranean Journal of Mathematics, 2016, DO
10.1007/s00009-016-0738-0

\bibitem{P44} {M. Berkani, M. Boudhief, N. Moalla,} { \em A characterization of unbounded generalized meromorphic
operators.}  FILOMAT, 32 (15) 2018.

\bibitem{CA-1}{S.R. Caradus}, {\em Operators of Riesz type}   Pacific Journal of Mathematics,
Vol.18, No.1, 1966.

\bibitem{DDHZ}{B.P. Duggal, D.S. Djordjevi\'c, R.E. Harte, S. \^C. \^Zivkovi\'c-Zlatanovi\'c} { \em
Polynomially meromorphic operators ,} Mathematical Proceedings of
the Royal Irish Academy, 116 A (1), 71-86 (2016).

\bibitem{HEU} {H. Heuser, } {\em  Functional Analysis,} Wiley
Interscience, Chichester. 1982

\bibitem{TLY}{ A.E. Taylor, D.C. Lay,} { \em Introduction to functional analysis} Krieger publishing company, 1980.

\bibitem{WEST} { T.T. West,}
{ \em  The decomposition of linear operators}, Proc. London. Math.
Soc. (3) 16(1966) 737-752.

\end{thebibliography}

 \baselineskip=12pt
\bigskip
\vspace{-5 mm }
 \baselineskip=12pt
\bigskip

{\tiny
\noindent Mohammed Berkani,\\
 \noindent Department of Mathematics,\\
 \noindent Science faculty of Oujda,\\
\noindent University Mohammed I,\\
\noindent Laboratory LAGA, \\
\noindent Morocco\\
\noindent berkanimo@aim.com,\\




\end{document}
